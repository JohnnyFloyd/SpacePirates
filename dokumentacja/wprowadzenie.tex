\chapter{Wprowadzenie}

\section{Wstępny opis}
Space Pirates to gra będąca połączeniem kosmicznej strzelaniny z elementami RPG. Gracz kierujew niej statkiem kosmicznym, który na swojej drodze spotyka zarówno wrogie statki, tytułowych piratów - antagonistów, jak i przyjazne jednostki, mogące zlecić zadania. Wraz z wykonywaniem kolejnych misji i pokonywaniem większej ilości przeciwników, rośnie doświadczenie bohatera. Awansując na kolejne stopnie wojskowe, gra umożliwia rozwijanie statystyk statku i odblokowywanie nowych ulepszeń broni, pancerza itd. Celem gry jest oczyszczenie przestrzeni kosmicznej z piratów. Docelowa platforma to system Android.

\section{Tło fabularne}
Cieżkie i burzliwe czasy nastały dla Federacji. Po ostatniej wojnie domowej handel i podróże międzyplanetarne stały się dużo bardziej niebezpieczne, na dobytek i towary zwykłych obywateli czychają piraci i grupy przestępcze. Jako młody kadet, absolwent Akademii, zostałeś wysłany na odlegle rubieże galaktyki(i na głęboką wodę zarazem) by zaprowadzić porządek w przestrzeni kosmicznej. Od powodzenia twojej misji zależy bezpieczeństwo i dobrobyt mieszkańców układów Carbona!

\section{Świat gry}
Gra jest podzielona na poziomy. Świat nie jest liniowy tzn. by przejść do ostatniego etapu nie trzeba przechodzić przez wszystkie dotychczasowe. Ma on strukturę drzewiastą gdzie najpierw możliwa jest eksploracja jednej gałęzi kosmosu, a w miarę postępów w rozgrywce kolejne fragmenty wszechświata są otwarte na przeszukiwanie.

\section{Sterowanie}
Gracz kieruje statkiem kosmicznym, widzi go z tyłu w przestrzeni 3D. Cały czas porusza się do przodu, jedyne ruchy jakie może wykonać do poruszanie się w przestrzeni 2D : góra-dół, lewo-prawo. Cała reszta obiektów w grze, zbliża się do nas. Niektóre szybciej(przeciwnicy), inni wolniej(sprzymierzeńcy). Wynikiem tego jest sytuacja, w której jeśli nie zniszczymy jakiegoś wroga, ani on nie zabije nas, to gdy jest bardzo blisko, "zniknie" za nami i już nie możemy go spotkać. Jako, że konieczne jest usunięcie wszystkich wrogów z obszarów to gracz musi być czujny i dbać o to, żeby móc zniszczyć piratów zanim znikną za nim.

\section{Opis rozgrywki}
Można wyróżnić dwa podstawowe działania dla gracza: walka i wykonywanie zadań.

\subsection{Walka}
Celem gry jest pokonanie przeciwników, tak więc centrum gry jest walczenie z piratami. Na początku, będą to słabe statki, ale wraz z odkrywaniem kolejnych obszarów, wróg będzie liczniejszy, twardszy i silniejszy. Gracz będzie musiał stosować taktykę Hit \& Run tzn. ostrzeliwać przeciwnika i równocześnie unikać jego ataków. Wraz z momentem pokonania ostatniego przeciwnika w konkretnym obszarze, pojawi się informacja o zaliczeniu tego poziomu. Wtedy, gracz będzie mógł ponownie przebyć ten obszar automatycznie, chyba, że zechce spotkać się z jednostką przebywająca na danym obszarze. W przypadku, gdy nie uda się pokonać wszystkich wrogów za jednym razem, przy ponownym wejściu do obszaru trzeba będzie stoczyć znów wszystkie walki, przy czym za każdym razem rozmieszenie jest losowane.

\subsection{Zadania}
Nie wszystkie jednostki w kosmosie są wrogo nastawione. Nie rzadko gracz spotka przyjaźnie nastawione statki handlowe, transportowe i wojskowe, które będą zlecać nam zadania. Wykonanie takie zadania, może być wynagrodzone odblokowaniem obszaru, nową bronią albo dodatkowymi punktami doświadczenia. Ukończenie gry nie wymaga wykonania wszystkich zadań, ale ułatwia to i urozmaica rozgrywkę.

\section{Rozwój postaci}
Każdy pirat jest warty punkty doświadczenia, które rosną wraz z jego siłą. Wykonanie zadań jest również punktowane. Gdy gracz zdobędzie odpowiednią liczbę punktów doświadczenia(od tego momentu nazywane PD) to awansuje na kolejny poziom w hierarchii wojskowej, a zaczyna ze stopnia szeregowego. Może wtedy kupić za punkty rozwoju przyznawane na każdym stopniu nowe bronie, pancerz, silniki albo kupić zupełnie nowy statek.