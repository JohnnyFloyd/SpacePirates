\chapter{Mechanika}

\section{Pojęcia}

Zasoby
\begin{itemize}
\item{ Punkty doświadczenia(PD) - miara mocy i umiejętności gracza, przyznawana za wykonywanie zadań i pokonanych przeciwników}
\item{Wytrzymałość - oznacza jak bardzo odporny na obrażenia jest dany statek(przeciwnik bądź gracz). Gdy stanie się mniejsze lub równe 0, statek eksploduje i znika. Nazywana jest też pancerzem.}
\item{Punkty rozwoju(PR) - waluta otrzymywana za awansowanie na kolejne poziomy doświadczenia, służąca do kupna ulepszeń statku}
\end{itemize}

Statek - jednostka w grze, zarówno gracz, przeciwnik jak i postać niezależna. Postacie niezależne nie posiadają w ogóle statystyk, za wyjątkiem przypadków kiedy jest to wytrzymałość.

Atrybuty statków
\begin{itemize}
\item{ Szybkość - prędkość poruszania się statku }
\item{Atak - ilość obrażeń, które zadaje pojedynczy strzał}
\item{Czas przeładowania - minimalny czas pomiędzy kolejnymi strzałami w sekundach}
\item{Skala wyzwania(SW) - miara trudności przeciwnika, zadania. Za każdego pokonanego wroga gracz otrzymuje punkty doświadczenia w zależności od poziomu gracza i SW właśnie}
\end{itemize}

Atrybuty gracza
\begin{itemize}
\item{ Poziom - doświadczenie gracza, które przekroczyło odpowiedni próg }
\item{Regeneracja - ile \% maksymalnej wytrzymałości statku jest odnawianych co sekundę}
\end{itemize}


\section{Gracz}
Statek bohatera porusza się cały czas do przodu. Możliwy jest ruch tylko w osi poziomej i pionowej.
Poniżej znajdują się podstawowe charakterystyki gracza, które są ustalane na początku rozgrywki. Awansując na kolejne poziomy, może poprawić te statystyki. Gdy życie gracza spadnie do 0 lub mniej, gracz kończy grę.

\begin{table}[h]
\centering
\begin{tabular}{ | c | c | }
\hline
\textbf{Nazwa} & \textbf{Wartość} \\ \hline
Wytrzymałość & 30 \\ \hline
Atak & 5 \\ \hline
Czas ładowania & 1 sekunda \\ \hline
Regeneracja & 0\% \\ \hline
Poziom & 1 \\ \hline
\end{tabular}
\caption{Początkowe statystyki}
\end{table}

\subsection{Zdobywanie doświadczenia}
Doświadczenie można zdobyć przez niszczenie wrogich statków oraz wykonywanie zadań. Gdy gracz zdobędzie odpowiednio dużo doświadczenia może wtedy awansować na kolejny poziom. Otrzymuje wtedy pewną liczbę punktów rozwoju, które może wydać na usprawnienia.

\begin{table}[h]
\centering
\begin{tabular}{ | c | c | c | }
\hline
\textbf{Poziom} & \textbf{PD do awansu} & \textbf{PR} \\
\hline
1 & 1000 & 3 \\
2 & 3000 & 3 \\
3 & 7000 & 5 \\
4 & 13000 & 5 \\
5 & 21000 & 7 \\
6 & 31000 & 7 \\
7 & 43000 & 7 \\
8 & 57000 & 9 \\
9 & 73000 & 9 \\
10 & 91000 & 11 \\
11 & 111000 & 11 \\
12 & 133000 & 13 \\
13 & 157000 & 13 \\
\hline

\end{tabular}
\caption{Liczba wymaganych PD do awansowania na każdym poziomie}
\end{table}

Wzór na to ile punktów doświadczenia gracz o poziome P by przejść na kolejny poziom:
\begin{align*}
EXP(1) = 1000 \\
EXP(P) = EXP(P-1) + (P-1)*2000 
\end{align*}

Wzór na liczbę punktów doświadczenia dla gracza o poziomie P i wyzwaniu o trudności równej SW:
\begin{align*}
\left \lfloor P*150*1.4^{SW-P} \right \rfloor
\end{align*}

\section{Przeciwnicy}
Przeciwnicy poruszają się w kierunku gracza i cały czas strzelają w jego stronę. Ich strzały uszkadzają wyłącznie statek gracza, a w momencie przypadkowego zetknięcia wrogiego statku z wrogimi pociskami, nic się nie dzieje. Przeciwnicy unikają statku gracza, ale jeśli nastąpi kolizja to oba są niszczone. Unikają również kolizji między własnymi statkami.

\begin{table}[h]
\centering
\begin{tabular}{ | c | c | c | c | c | c | }
\hline
\textbf{Nazwa} & \textbf{Wytrzymałość} & \textbf{Atak} & \textbf{Szybkość} & \textbf{Czas przeładowania} & \textbf{SW} \\ \hline
Myśliwiec 	& 25 	& 3 	& 1 	& 0.7	& 1		\\ \hline
Patrolowiec & 45 	& 5 	& 1 	& 0.7 	& 2 	\\ \hline
Korweta 	& 70 	& 10 	& 1 	& 1 	& 4 	\\ \hline
Fregata 	& 150 	& 20 	& 1 	& 1		& 6		\\ \hline
Niszczyciel & 300 	& 35 	& 0.7 	& 1		& 9 	\\ \hline
Krążownik 	& 500 	& 50 	& 0.5 	& 1.2	& 13	\\ \hline
Pancernik 	& 1000 	& 80 	& 0.4 	& 1.5	& 17	\\ \hline
\end{tabular}
\caption{Statystyki wrogich statków}
\end{table}

\section{Usprawnienia}
Poniżej znajdują się tabele z usprawnieniami dla broni, pancerza i silników opisujące koszt danego usprawnienia, jego wpływ na postać i jakie usprawnienia należy wcześniej wykupić by mieć dostęp do niego. Można je nabywać wyłącznie w punktach kontrolnych między obszarami.

\begin{table}[h]
\centering
\begin{tabular}{ | c | c | c | c | }
\hline
\textbf{Nazwa} & \textbf{Koszt w PR} & \textbf{Działanie} & \textbf{Wymagania}  \\
\hline
Miotacz Laserowy I & 3 & Siła ataku wynosi 7 & \\ \hline
Miotacz Laserowy II & 5 & Siła ataku wynosi 10 & Miotacz Laserowy I \\ \hline
Pociski Jonowe I & 10 & Siła ataku wynosi 15 & \\ \hline
Pociski Jonowe II & 15 & Siła ataku wynosi 25 & Pociski Jonowe I \\ \hline
Akcelerator cząstek & 15 & Zmniejsza czas ładowania do 0.8 s & \\ \hline
Pierścienie Plancka & 20 & Zmniejsza czas ładowania do 0.5 s & \\ \hline
Działo Higgsa & 30 & Siła ataku wynosi 40 & \\ \hline
\end{tabular}
\caption{Tablica usprawnień dla broni}
\end{table}

\begin{table}[h]
\centering
\begin{tabular}{ | c | c | c | }
\hline
\textbf{Nazwa} & \textbf{Koszt w PR} & \textbf{Opis} \\
\hline
Kompozyty & 5 & Wytrzymałość wynosi 45 \\ \hline
Mikrowłókna & 10 & Wytrzymałość wynosi 70 \\ \hline
Nanowłókna & 15 & Wytrzymałość wynosi 120 \\ \hline
Reperator & 20 & Wytrzymałość odbudowuje się z prędkością 1.5\% sekundę \\ \hline
Lekki pancerz tytanowy & 20 & Wytrzymałość wynosi 180 \\ \hline
Ciężki pancerz tytanowy & 30 & Wytrzymałość wynosi 300 \\ \hline
\end{tabular}
\caption{Tablica usprawnień dla pancerza}
\end{table}

\begin{table}[h]
\centering
\begin{tabular}{ | c | c | c | }
\hline
\textbf{Nazwa} & \textbf{Koszt w PR} & \textbf{Opis} \\
\hline
Napęd Fuzyjny & 15 & Prędkość rośnie o 25\% \\ \hline
Hipernapęd & 30 & Prędkość rośnie o 60\% \\ \hline
\end{tabular}
\caption{Tablica usprawnień dla silników}
\end{table}
