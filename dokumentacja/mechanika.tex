\chapter{Mechanika}

\section{Gracz}
Statek bohatera porusza się cały czas do przodu. Możliwy jest ruch tylko w osi poziomej i pionowej.
Poniżej znajdują się podstawowe charakterystyki gracza, które są ustalane na początku rozgrywki. Awansując na kolejne poziomy, może poprawić te statystyki.

\begin{table}[h]
\centering
\begin{tabular}{ | c | c | p{6cm} | }
\hline
\textbf{Nazwa} & \textbf{Wartość} & \textbf{Opis} \\ \hline
Pancerz gracza & 30 & Wytrzymałość statku gracza \\ \hline
Siła ataku & 5 & Ilość obrażeń zadawanych każdym trafieniem przeciwnikowi\\ \hline
Czas ładowania & 1 sekunda & Minimalny odstęp między kolejnymi wystrzałami\\ \hline
\end{tabular}
\caption{Początkowe statystyki gracza}
\end{table}

\section{Zdobywanie doświadczenia}
Doświadczenie można zdobyć przez niszczenie wrogich statków oraz wykonywanie zadań. Gdy gracz zdobędzie odpowiednio dużo doświadczenia może wtedy awansować na kolejny poziom. Otrzymuje wtedy pewną liczbę punktów rozwoju(od tego momentu nazywane PK), które może wydać na usprawnienia.

\begin{table}[h]
\centering
\begin{tabular}{ | c | c | c | }
\hline
\textbf{Poziom} & \textbf{PD do awansu} & \textbf{PK} \\
\hline
1 & 1000 & 3 \\
2 & 3000 & 3 \\
3 & 7000 & 5 \\
4 & 13000 & 5 \\
5 & 21000 & 7 \\
6 & 31000 & 7 \\
7 & 43000 & 7 \\
8 & 57000 & 9 \\
9 & 73000 & 9 \\
10 & 91000 & 10 \\
11 & 111000 & 12 \\
12 & 133000 & 12 \\
13 & 157000 & 15 \\
\hline

\end{tabular}
\caption{Liczba wymaganych PD do awansowania na każdym poziomie}
\end{table}

\section{Usprawnienia}
Poniżej znajdują się tabele z usprawnieniami dla broni, pancerza i silników opisujące koszt danego usprawnienia, jego wpływ na postać i jakie usprawnienia należy wcześniej wykupić by mieć dostęp do niego.